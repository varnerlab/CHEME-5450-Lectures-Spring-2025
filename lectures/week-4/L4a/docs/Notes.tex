\documentclass{article}[11pt]
\usepackage{fullpage,graphicx, setspace, latexsym, cite,amsmath,amssymb,xcolor,subfigure}
%\usepackage{epstopdf}
%\DeclareGraphicsExtensions{.pdf,.eps,.png,.jpg,.mps} 
\usepackage{amssymb} %maths
\usepackage{amsmath} %maths
\usepackage{amsthm, comment}
\usepackage[round,comma,sort,numbers]{natbib}

% \bibliographystyle{plain}
\bibliographystyle{plos2015}

\newtheorem{theorem}{Theorem}
\newtheorem{prop}{Proposition}
\newtheorem{corollary}{Corollary}
\newtheorem{lemma}{Lemma}
\newtheorem{defn}{Definition}
\newtheorem{ex}{Example}
\usepackage{float}

\newcommand*{\underuparrow}[1]{\underset{\uparrow}{#1}}
\usepackage{graphicx}
\usepackage{xcolor}
\usepackage[dvipsnames]{xcolor}
\usepackage{algorithmicx}
\usepackage{algorithm} %http://ctan.org/pkg/algorithms
\usepackage{algpseudocode} %http://ctan.org/pkg/algorithmicx
\usepackage{enumitem}
\usepackage{simplemargins}
\usepackage{hyperref}

\usepackage{mdframed}
\definecolor{lgray}{rgb}{0.92,0.92,0.92}
\definecolor{lsalmon}{rgb}{0.9921568627450981,0.9411764705882353, 0.9254901960784314}

\renewcommand{\bibnumfmt}[1]{#1.}
\setlist{noitemsep} % or \setlist{noitemsep} to leave space around whole list
\setallmargins{1in}
\linespread{1.1}

\newcommand{\brows}[1]{%
  \begin{bmatrix}
  \begin{array}{@{\protect\rotvert\;}c@{\;\protect\rotvert}}
  #1
  \end{array}
  \end{bmatrix}
}
\newcommand{\rotvert}{\rotatebox[origin=c]{90}{$\vert$}}
\newcommand{\rowsvdots}{\multicolumn{1}{@{}c@{}}{\vdots}}


\def\R{\mathbb{R}}
\def\Eps{\mathcal{E}}
\def\E{\mathbb{E}}
\def\V{\mathbb{V}}
\def\F{\mathcal{F}}
\def\G{\mathcal{G}}
\def\H{\mathcal{H}}
\def\S{\mathcal{S}}
\def\D{\mathcal{D}}
\def\P{\mathbb{P}}
\def\1{\mathbf{1}}
\def\n{\nappa}
\def\h{\mathbf{w}}
\def\v{\mathbf{v}}
\def\x{\mathbf{x}}
\def\X{\mathcal{X}}
\def\Y{\mathcal{Y}}
\def\eps{\epsilon}
\def\y{\mathbf{y}}
\def\e{\mathbf{e}}
\newcommand{\norm}[1]{\left|\left|#1\right|\right|}
\DeclareMathOperator*{\argmin}{arg\,min}
\DeclareMathOperator*{\argmax}{arg\,max}
\newcommand{\lecture}[4]{
   \pagestyle{myheadings}
   \thispagestyle{plain}
   \newpage
   % \setcounter{lecnum}{#1}
   \setcounter{page}{1}
   \setlength{\headsep}{10mm}
   \noindent
   \begin{center}
   \framebox{
      \vbox{\vspace{2mm}
    \hbox to 6.28in { {\bf CHEME 5820: Machine Learning for Engineers
   \hfill Spring 2025} }
       \vspace{4mm}
       \hbox to 6.28in { {\Large \hfill Lecture #1: #2  \hfill} }
       \vspace{2mm}
       \hbox to 6.28in { {\it Lecturer: #3 \hfill #4} }
      \vspace{2mm}}
   }
   \end{center}
   \markboth{Lecture #1: #2}{Lecture #1: #2}

   \noindent{\bf Disclaimer}: {\it These notes have not been subjected to the
   usual scrutiny reserved for formal publications. }
   \vspace*{4mm}
}

\begin{document}
\lecture{4a}{Metabolic Networks and the Stoichiometric Matrix}{Jeffrey Varner}{}

\begin{mdframed}[backgroundcolor=lgray]
    In this lecture, we will discuss the following topics:
    \begin{itemize}[leftmargin=16pt]
        \item{Introduction to metabolic networks}
        \item{The stoichiometric matrix}
        \item{The rank of the stoichiometric matrix}
        \item{The null space of the stoichiometric matrix}
    \end{itemize}
 \end{mdframed}

\section{Introduction}
In this lecture we'll introduce metabolic networks and the stoichiometric matrix, which is a fundamental tool in systems biology for modeling and analyzing biochemical reaction systems.
Metabolic networks are complex systems of enzyme catalyzed biochemical reactions that convert nutrients into energy and building blocks for cellular components, or products of interest.
These chemical reaction networks can be mathematically represented using stoichiometric matrices. 
A stoichiometric matrix provides a systematic way to represent the flow of material through various biochemical reactions, where rows correspond to chemical species (metabolites) and columns correspond to reactions. 
This matrix is essential for constraint-based modeling and flux balance analysis, allowing researchers to analyze and predict metabolic behavior under different conditions. 
Today, we'll introduce the stoichiometric matrix and discuss its properties, including rank and null space.

\section{Stoichiometric Matrix}
Suppose we have a set of chemical reactions $\mathcal{R}$ involving the chemical species (metabolite) set $\mathcal{M}$ occurring in some volume $V$.
This can be a physical volume, such as a test tube, or a biological volume, such as a cell or even a logical volume, such as a compartment in a model.
We can represent this metabolic network using the stoichiometric matrix $\mathbf{S}$ (Defn \ref{defn-stoichiometric-matrix}):

\begin{defn}[Stoichiometric Matrix]\label{defn-stoichiometric-matrix}
The stoichiometric matrix is a $\mathbf{S}\in\mathbb{R}^{|\mathcal{M}|\times|\mathcal{R}|}$ matrix that holds the stoichiometric coefficients $\sigma_{ij}\in\mathbf{S}$ such that:
   \begin{itemize}[leftmargin=16pt]
      \item{$\sigma_{ij}>0$: Chemical species $i$ is \textit{produced} by reaction $j$. Species $i$ is a product of reaction $j$.}
      \item{$\sigma_{ij} = 0$: Chemical species $i$ is not connected with reaction $j$}
      \item{$\sigma_{ij}<0$: Chemical species $i$ is \textit{consumed} by reaction $j$. Species $i$ is a reactant of reaction $j$.}
   \end{itemize}
   The stoichiometric matrix is a digital representation of the metabolic network, and it can be used to analyze the structure of the network, 
   as well as to estimate operational values, such as the metabolic fluxes.
\end{defn}

The stoichiometric matrix $\mathbf{S}$ is the digital representation of the metabolic network, and it can be used to analyze some of the structural properties of the network.
There is a large body of literature on the analysis of the structure of metabolic networks using various decomposition approaches of stoichiometric matrices. 
Let's look at a few of these, starting with the connectivity of the network, the rank of the stoichiometric matrix, 
and finally the decomposition using singular value decomposition (SVD) \cite{Famili:2003aa}.

\section{Connectivity of the Network}
The connectivity of a metabolic network, and the connectivity distribution, gives us information about the network's structure, e.g., 
the number of reactions that are connected to a given metabolite, or the number of metabolites that are connected to a given reaction, etc.
We compute the connectivity of the network using the binary stoichiometric matrix $\bar{\mathbf{S}}$ (Defn. \ref{defn-binary-stoichiometric-matrix}).

\begin{defn}[Binary stoichiometric Matrix]\label{defn-binary-stoichiometric-matrix}
Suppose we have a stoichiometric matrix $\mathbf{S}$, the binary stoichiometric matrix $\bar{\mathbf{S}}$ is a matrix with the same dimensions as $\mathbf{S}$, 
where the elements are binary, i.e., $\bar{\sigma}_{ij} = 1$ if $\sigma_{ij}\neq 0$, and $\bar{\sigma}_{ij} = 0$ if $\sigma_{ij} = 0$.
Matrix products of $\bar{\mathbf{S}}$ with its transpose $\bar{\mathbf{S}}^{\top}$ (or vice-versa) give us connectivity information about the network.
\end{defn}

\subsection{Metabolite connectivity}
Fill me in.

\subsection{Reaction connectivity}
Fill me in.

\section{Rank of the Stoichiometric Matrix}
The rank of the stoichiometric matrix is a key property that provides information about the network's structure and the number of independent reactions.
We can compute the rank (and get some other useful information) using singular value decomposition (SVD) of the stoichiometric matrix.
In the context of the singular value decomposition, the rank of the stoichiometric matrix is the number of non-zero singular values, 
or equvialently, the number of unique information modes contained in the matrix. 
Let's take a look at the singular value decomposition of a general rectangular matrix $\mathbf{A}$, 
and then apply SVD to the stoichiometric matrix and discuss its properties.

\subsection{Singular Value Decomposition (SVD)}
Singular value decomposition (SVD), originally developed in the 1870s \citep{Stewart:1993} is a matrix factorization technique that is based on the eigendecomposition of a matrix.
Suppose we have a matrix $\mathbf{A} \in \R^{m \times n}$. The SVD of $\mathbf{A}$ is a factorization of the form: $\mathbf{A} = \mathbf{U}\mathbf{\Sigma}\mathbf{V}^{\top}$, where
$\mathbf{U}\in\mathbb{R}^{n\times{n}}$ and $\mathbf{V}\in\mathbb{R}^{m\times{m}}$ are orthogonal matrices, i.e., $\mathbf{U}\cdot\mathbf{U}^{\top} = \mathbf{I}$ and $\mathbf{\Sigma}\in\mathbb{R}^{n\times{m}}$ is a diagonal matrix containing 
the singular values $\sigma_{i}=\Sigma_{ii}$. The matrix $\mathbf{A}\in\mathbb{R}^{n\times{m}}$ can be decomposed as:
\begin{equation}
\mathbf{A} = \sum_{i=1}^{r_{\mathbf{A}}}\sigma_{i}\cdot\left(\mathbf{u}_{i}\otimes\mathbf{v}_{i}\right)
\end{equation}
where $r_{\mathbf{A}}$ is the rank of matrix $\mathbf{A}$, and $\sigma_{i}$ are the singular values (ordered from largest to smallest) of the matrix $\mathbf{A}$,
and $\otimes$ denotes the outer product. 
The outer product $\hat{\mathbf{A}}_{i} = \mathbf{u}_{i}\otimes\mathbf{v}_{i}$ is a rank-1 matrix, i.e., a mode of the original matrix $\mathbf{A}$,  with elements: 
\begin{equation}
\hat{a}_{jk} = u_{j}v_{k}\qquad{j=1,2,\dots,n~\text{and}~k=1,2,\dots,m}
\end{equation}
where the vectors $\mathbf{u}_{i}$ and $\mathbf{v}_{i}$ denote the left (right) singular vectors, respectively, of the matrix $\mathbf{A}$.
Singular value decomposition is a powerful tool for analyzing the structure of a matrix, e.g., for computing properties such as the rank of a matrix, and is used in many applications, 
including data compression, image processing, and control theory. It is also used for solving linear systems of equations,
such as those that arise in linear regression tasks. The SVD is also used in a huge variety of unsupervised learning type applications, e.g., understanding gene expression data \citep{Alter:2000aa, Alter:2006},  
the structure of chemical reaction networks \citep{Famili:2003aa}, 
in process control applications \citep{MooreSVD1986}, and analysis of various type of human centered networks \citep{SASTRY20075275, 7993780}.


\bibliography{References-W4.bib}

\end{document}