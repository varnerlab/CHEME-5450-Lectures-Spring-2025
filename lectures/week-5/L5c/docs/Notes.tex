\documentclass{article}[11pt]
\usepackage{fullpage,graphicx, setspace, latexsym, cite,amsmath,amssymb,xcolor,subfigure}
%\usepackage{epstopdf}
%\DeclareGraphicsExtensions{.pdf,.eps,.png,.jpg,.mps} 
\usepackage{amssymb} %maths
\usepackage{amsmath} %maths
\usepackage{amsthm, comment}
\usepackage[round,comma,sort,numbers, compress]{natbib}

% \bibliographystyle{plain}
\bibliographystyle{plos2015}

\newtheorem{theorem}{Theorem}
\newtheorem{prop}{Proposition}
\newtheorem{corollary}{Corollary}
\newtheorem{lemma}{Lemma}
\newtheorem{defn}{Definition}
\newtheorem{ex}{Example}
\usepackage{float}

\newcommand*{\underuparrow}[1]{\underset{\uparrow}{#1}}
\usepackage{graphicx}
\usepackage{xcolor}
\usepackage[dvipsnames]{xcolor}
\usepackage{algorithmicx}
\usepackage{algorithm} %http://ctan.org/pkg/algorithms
\usepackage{algpseudocode} %http://ctan.org/pkg/algorithmicx
\usepackage{enumitem}
\usepackage{simplemargins}
\usepackage{hyperref}
\hypersetup{
     colorlinks=true,
     linkcolor=blue,
     filecolor=red,
     citecolor = red,      
     urlcolor=cyan,
}

\usepackage{mdframed}
\definecolor{lgray}{rgb}{0.92,0.92,0.92}
\definecolor{lsalmon}{rgb}{0.9921568627450981,0.9411764705882353, 0.9254901960784314}

\renewcommand{\bibnumfmt}[1]{#1.}
\setlist{noitemsep} % or \setlist{noitemsep} to leave space around whole list
\setallmargins{1in}
\linespread{1.1}

\newcommand{\brows}[1]{%
  \begin{bmatrix}
  \begin{array}{@{\protect\rotvert\;}c@{\;\protect\rotvert}}
  #1
  \end{array}
  \end{bmatrix}
}
\newcommand{\rotvert}{\rotatebox[origin=c]{90}{$\vert$}}
\newcommand{\rowsvdots}{\multicolumn{1}{@{}c@{}}{\vdots}}


\def\R{\mathbb{R}}
\def\Eps{\mathcal{E}}
\def\E{\mathbb{E}}
\def\V{\mathbb{V}}
\def\F{\mathcal{F}}
\def\G{\mathcal{G}}
\def\H{\mathcal{H}}
\def\S{\mathcal{S}}
\def\D{\mathcal{D}}
\def\P{\mathbb{P}}
\def\1{\mathbf{1}}
\def\n{\nappa}
\def\h{\mathbf{w}}
\def\v{\mathbf{v}}
\def\x{\mathbf{x}}
\def\X{\mathcal{X}}
\def\Y{\mathcal{Y}}
\def\eps{\epsilon}
\def\y{\mathbf{y}}
\def\e{\mathbf{e}}
\newcommand{\norm}[1]{\left|\left|#1\right|\right|}
\DeclareMathOperator*{\argmin}{arg\,min}
\DeclareMathOperator*{\argmax}{arg\,max}
\newcommand{\lecture}[4]{
   \pagestyle{myheadings}
   \thispagestyle{plain}
   \newpage
   % \setcounter{lecnum}{#1}
   \setcounter{page}{1}
   \setlength{\headsep}{10mm}
   \noindent
   \begin{center}
   \framebox{
      \vbox{\vspace{2mm}
    \hbox to 6.28in { {\bf CHEME 5820: Machine Learning for Engineers
   \hfill Spring 2025} }
       \vspace{4mm}
       \hbox to 6.28in { {\Large \hfill Lecture #1: #2  \hfill} }
       \vspace{2mm}
       \hbox to 6.28in { {\it Lecturer: #3 \hfill #4} }
      \vspace{2mm}}
   }
   \end{center}
   \markboth{Lecture #1: #2}{Lecture #1: #2}

   \noindent{\bf Disclaimer}: {\it These notes have not been subjected to the
   usual scrutiny reserved for formal publications. }
   \vspace*{4mm}
}

\begin{document}
\lecture{5c}{Introduction to the Constraints in Flux Balance Analysis}{Jeffrey Varner}{}

\begin{mdframed}[backgroundcolor=lgray]
    In this lecture, we will discuss the following topics:
    \begin{itemize}[leftmargin=16pt]
    \item{\textbf{Metabolism and metabolic networks}: A metabolic network is the complete set of metabolic (chemical) processes determining a cell's biochemical state. It encompasses all the chemical reactions associated with metabolism, i.e., the breakdown of raw materials such as sugars (catabolism) and the production of macromolecules, e.g., proteins, lipids, etc (anabolism).}
    \item{\textbf{A stoichiometric matrix} is a mathematical representation of a metabolic network that encodes the relationships between reactants and products in the network, where rows correspond to different metabolites, contained in the set $\mathcal{M}$, and columns correspond to reactions, contained in the set $\mathcal{R}$. Thus, the stoichiometric matrix is a $\mathbf{S}\in\mathbb{R}^{|\mathcal{M}|\times|\mathcal{R}|}$ matrix holding the stochiometric coefficients $\sigma_{ij}\in\mathbf{S}$ for $i=1,2,\dots,|\mathcal{M}|$ and $j=1,2,\dots,|\mathcal{R}|$.}
    \item{\textbf{Structural analysis of $\mathbf{S}$}: Structural analysis of the stoichiometric matrix involves examining its connectivity distribution and using tools such as eigendecomposition to explore the network's fundamental pathway structures and other topological properties. These types of analyses give us more insight into the structure of the network (and perhaps some indication of the importance of particular metabolites of reactions).}
    \end{itemize}
 \end{mdframed}

\section{Introduction}

\bibliography{References-W4.bib}

\end{document}